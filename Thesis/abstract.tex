\cleardoublepage
%\phantomsection
%\addcontentsline{toc}{chapter}{Sommario}
\chapter*{Sommario}
\par Lo scopo della presente tesi � di illustrare l'applicazione di tecniche di riduzione della complessit� del modello di radiazione di antenne a schiera fasate, simulate con approccio \itt{full-wave} mediante la tecnica degli elementi finiti. 
\par Si sviluppa, inizialmente, la formulazione matematica che permette la trasformazione da campo vicino a campo lontano, il campo vicino essendo valutato mediante la tecnica degli elementi finiti in un dominio chiuso. Segue un'analisi approfondita del comportamento del campo vicino a fronte di variazioni della direzione di scansione, necessaria per prevedere l'entit� della riduzione di complessit�. Vengono inoltre presentate tecniche di interpolazione polinomiale del funzionale di trasformazione da campo vicino a campo lontano. Infine, il sistema lineare del modello di radiazione, derivato da una formulazione di Galerkin-elementi finiti, viene parametrizzato nella direzione di scansione e in quella di osservazione, operazione che ne permette una riduzione sistematica della complessit�. Esempi numerici illustrano il grado di accuratezza ottenuto nella computazione del campo lontano in seguito alle approssimazioni eseguite.

%\cleardoublepage
%\phantomsection
%\addcontentsline{toc}{chapter}{Abstract}
%\chapter*{Abstract}
%\par 
