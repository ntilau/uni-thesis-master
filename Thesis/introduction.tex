\cleardoublepage
\renewcommand\thefigure{\arabic{figure}}
\markright{INTRODUCTION}
\pagestyle{myheadings}

\phantomsection
\addcontentsline{toc}{chapter}{Introduction}
\chapter*{Introduction}

\par Phased array antennas consist of a stationary spatial disposition of multiple coherently fed antennas that, acting on the phase of the excitations, allow to control the direction of the maximum gain achieved by the whole radiating structure. The array radiation pattern, as a result of far fields constructive and destructive interferences, is dependent on the relative phase shifts between the far fields produced by the antennas or \itt{array elements}. An additional phase shift in the excitations simply modifies the directions of the constructive and destructive interferences, allowing a \itt{beam steering} operation. Also, variable amplitude control of the excitations is often provided for pattern shaping. \cite{BalanisAT, StutzmanThieleATD, MilliganMAD, SelleriETA, MaillouxPAAH}

\par These radiating structures are of a great importance in modern radar and communication systems. For example, modern fire control and artillery location radars make use of phased array antennas instead of mechanical beam steering systems, the latter ones having higher reaction times in target tracking and lower times to failure than mechanically fixed ones. Also, current cellular communication systems require, for the base stations, the ability to modify the radiation pattern in order to fit the illuminated zones to the desired ones and, if necessary, to hide undesired interferers by \itt{null-steering}, that is the synthesis of a destructive interference in the direction of the interferer. These so-called \itt{smart antennas} \cite{ZooghbySAE} are currently employed in the modern spatial diversity based communications techniques as the third generation cellular systems.

\par An accurate prediction of array parameters using numerical methods not only reduces the development cost and design period but also renders invaluable information to design engineers. For this purpose, several techniques like the method of moments (MoM), the finite-element method (FEM) and finite-difference time-domain (FDTD) are available. To accurately model the electromagnetic field behavior inside each array element and the mutual coupling between the elements, a three dimensional full-wave analysis is necessary. Furthermore, an accurate analysis of radiation problems is impossible without accurate modeling of the feeding structure. Due to its general formulation and versatility in geometrical modeling, the FEM appears to be the most suitable technique as an analysis tool. 

\par However, the FEM requires very large computational and memory resources to allow fast design process. This is particularly true when the number of input parameters to consider in an effective design is high, and the simulations have to be iterated for each parameters set. Several techniques have been introduced in the last two decades to face this problem. One of them, which will be presented in this thesis, splits the computations in two steps. The first, an \itt{off-line} stage  somewhat time and resources consuming but that needs to be run only once, provides an accurately chosen approximated model of the electromagnetic structure analyzed. The second, an \itt{on-line} stage, allows a fast computation of the sought informations in front of variable input parameters, with an error dependent on the approximations made in the first step. These techniques are referred to as the \itt{model order reduction} (MOR) techniques, and they have been successfully employed in multi-parameters wave guiding and scattering problems \cite{SelleriEdlingerDDMOR,BertazziGhioneMOR, FarleMORFE, LedgerPerairePESSRIWA}.

\par The FEM analysis of antenna arrays is typically limited to the near fields surrounding the radiating structure, closing the domain with opportune boundary conditions on an encompassing surface. Thus, near fields to far fields transformations are necessary in order to derive the far fields pattern.

\par The purpose of the present thesis is to discuss the application of projection-based MOR techniques on frequency domain FEM simulations of planar phased arrays, leaving as inputs the excitations of the array elements and rapidly computing the far fields pattern in front of a beam steering process. In the first chapter, the electromagnetic radiation basics in a bounded medium will be reviewed, and the near field to far field transformation will be presented for both scalar and vector fields. Then, in chapter 2, the process of finding a suitable approximation space for the near field to far field transformation will be discussed. Finally, in chapter 3, the radiation model of phased arrays parameterized in the scan and look angles will be constructed, then a projection-based MOR technique will be applied. Two numerical examples validate the approximations performed in the second chapter, and the reduced order models will be compared to the respective full models.

\pagestyle{headings}
\renewcommand\thefigure{\thechapter.\arabic{figure}}
