%%%%% MY COMMANDS
%\newenvironment{frcseries}{\fontfamily{frc}\selectfont}{}
%\newcommand{\textfrc}[1]{{\frcseries#1}}
%\DeclareMathAlphabet{\mathbf}{T1}{ptm}{bx}{n}
%\DeclareMathAlphabet{\mathpzc}{OT1}{pzc}{m}{it}
\DeclareMathAlphabet{\mcal}{OMS}{cmbsy10}{b}{it}
\DeclareMathAlphabet{\mcalsf}{OMS}{cmmib10}{m}{it}

\newcommand{\quotes}[1]{``#1''}
\newcommand{\itt}[1]{\textit{#1}}
\newcommand{\mrm}[1]{\mathrm{#1}}
\newcommand{\itm}[1]{\mathit{#1}}
\newcommand{\mbf}[1]{\mathbf{#1}}
\newcommand{\mat}[1]{\mathbf{#1}}
\newcommand{\vect}[1]{\lowercase{ \mathbf{#1}}}
\newcommand{\dyad}[1]{\overline{\overline{#1}}}
\newcommand{\mbfit}[1]{\mathbold{#1}}
\newcommand{\eqref}[1]{(\ref{#1})}
%\newcommand{\varepsilonup}[0]{\varepsilon}
%\newcommand{\phiup}[0]{\phi}
%\newcommand{\iint}[1]{\int \!\!\!\!\! \int#1}
%\newcommand{\iiint}[1]{\int \!\!\!\!\! \int \!\!\!\!\! \int#1}
\newcommand{\�}[0]{\textrm{\textdegree}}
\newcommand{\spanning}[1]{ \mathrm{span} \left \{ #1 \right \} }
\newcommand{\colspan}[1]{ \mathrm{colsp} \left \{ #1 \right \} }
\newcommand{\kernel}[1]{ \mathrm{ker} \left \{ #1 \right \} }
\newcommand{\rank}[1]{ \mathrm{rank} \left ( #1 \right ) }
%\DeclareMathOperator{\rank}{rank}
%\DeclareMathOperator{\esssup}{ess\,sup}
\hyphenation{iso-tro-pic Max-well de-ri-va-to}

\newtheorem{mydef}{Definition}
\newtheorem{mythm}{Theorem}
\newtheorem{mycc}{Conclusion}


%\definecolor{mycolor}{rgb}{.122,.286,.490}
\definecolor{mycolor}{rgb}{.400,.400,.400}

%%%%%
\usepackage{float} % used to fix figures references
%%%%% PDF HYPERREF SETUP
\usepackage[backref=page]{hyperref}%[colorlinks]
\usepackage{memhfixc}
\usepackage[figure,table]{hypcap}
\hypersetup{ 
	pdfstartview={FitV}, % { FitH, FitV, XYZ null null 1}
	pdfpagelayout=OneColumn, % SinglePage OneColumn TwoColumnRight
	pdfdirection=L2R,
	pdfpagemode = {UseOutlines}, % UseNone, UseThumbs, UseOutlines, FullScreen, UseOC, UseAttachments
	pdfpagetransition=Blinds,
	linktocpage=true,
	colorlinks=true,
	linkcolor = mycolor,
	urlcolor= black,
	anchorcolor = mycolor,
	citecolor = mycolor,
	filecolor = mycolor,
	pdftitle = {LN-Thesis},
	pdfsubject = {Model Order Reduction in Finite Element Analysis of Phased Arrays},
	pdfkeywords = {Model Order Reduction, Finite Element Method, Near to Far Fields Transformations, Phased Arrays},
	pdfauthor = {Laurent Ntibarikure},
	pdfproducer = {Laurent Ntibarikure},
	pdfcenterwindow = true,
	pdfdisplaydoctitle = true,
	bookmarksopen=true,
	bookmarksnumbered,
	bookmarksopenlevel=-1, % \maxdimen 
	CJKbookmarks=true,
	bookmarkstype=toc,
	pdffitwindow=false,
	nesting=false,
	raiselinks=true,
	%hyperfootnotes=false
}

%%%%% Page layout
%\parindent 0pt
\parskip 1ex
\renewcommand{\baselinestretch}{1.5}
%\numberwithin{equation}{section}
%\renewcommand{\bibname}{Bibliografia}
%\renewcommand{\contentsname}{Indice}
\pagenumbering{roman}
\bibliographystyle{unsrtnat}
\renewcommand{\theequation}{\thesection.\arabic{equation}}
%\renewcommand{\thesection}{\arabic{section}}
%\renewcommand{\thesubsection}{(\arabic{subsection})}
%\renewcommand{\thesubsubsection}{(\arabic{subsubsection})}
\setcounter{tocdepth}{2}
\setcounter{secnumdepth}{3}
\setcounter{section}{0}
%\setcounter{figure}{0}
%\renewcommand\thefigure{\arabic{figure}}
%%%%%

%%%%% DESIGNS THE BOX CHAPTER NBR
\makeatletter
\newlength\dlf@normtxtw
\setlength\dlf@normtxtw{\textwidth}
\def\myhelvetfont{\def\sfdefault{mdput}}
\newsavebox{\feline@chapter}
\newcommand\feline@chapter@marker[1][4cm]{%
\sbox\feline@chapter{%
\resizebox{!}{#1}{\fboxsep=1pt%
\colorbox{mycolor}{\color{white}\bfseries\rmfamily\thechapter}%
}}%
%% inizio scritta capitolo
\rotatebox{90}{%
\resizebox{%
\heightof{\usebox{\feline@chapter}}+\depthof{\usebox{\feline@chapter}}}%
{!}{\scshape\so\@chapapp}}
\quad%
%% fine scritta capitolo
\raisebox{\depthof{\usebox{\feline@chapter}}}{\usebox{\feline@chapter}}%
}
\newcommand\feline@chm[1][4cm]{%
\sbox\feline@chapter{\feline@chapter@marker[#1]}%
\makebox[0pt][l]{% aka \rlap
\makebox[1cm][r]{\usebox\feline@chapter}%
}}
\makechapterstyle{mychapstyle}{
\renewcommand\chapnamefont{\normalfont\Large\scshape\raggedright\so}
\renewcommand\chaptitlefont{\normalfont\huge\bfseries\scshape\color{mycolor}}
\renewcommand\chapternamenum{}
\renewcommand\printchaptername{}
\renewcommand\printchapternum{\null\hfill\feline@chm[2.5cm]\par}
\renewcommand\afterchapternum{\par\vskip\midchapskip}
\renewcommand\printchaptertitle[1]{\chaptitlefont\raggedright ##1\par}
}
\makeatother
\chapterstyle{mychapstyle}
%%%%%

%%%%% FIX HYPERREF FOOTNOTELINKS Raising references
\makeatletter
\newcounter{Hfootnote}
\let\H@@footnotetext\@footnotetext
\let\H@@footnotemark\@footnotemark
\long\def\@footnotetext#1{%
  \H@@footnotetext{
  \Hy@raisedlink{\hyper@anchorstart{\@currentHref}\hyper@anchorend}{#1}% Inalza il link a pi� di pagina
 }%
}
\def\@footnotemark{%
  \leavevmode
  \ifhmode\edef\@x@sf{\the\spacefactor}\nobreak\fi
  \H@refstepcounter{Hfootnote}%
  \hyper@makecurrent{Hfootnote}%
  \hyper@linkstart{link}{\@currentHref}%
  \@makefnmark
  \hyper@linkend
  \ifhmode\spacefactor\@x@sf\fi
  \relax
}
\def\addhyperlinkline#1#2{%
  \global\advance\OddToc by 1
  % If we're in vmode we want to revert to vmode
  \edef\@tempa{\ifvmode\vskip0pt\fi}%
  \Hy@raisedlink{\hyper@@anchor{toc\the\OddToc}{\relax}\@tempa}%
  \@writetorep{}{#2}{toc\the\OddToc}{\csname toclevel@#1\endcsname}%
}
\makeatother
%%%%%

%%%%% FIX EQUATIONS REFERENCES Raising references
\makeatletter
\def\eqnarray{%
  \let\Hy@reserved@a\relax
  \def\@currentHref{}%
  \H@eqnarray
  \if@eqnstar
  \else
    \ifx\\\@currentHref\\%
    \else
      \@ifundefined{theHequation}{%
        \make@stripped@name{\theequation}%
        \let\theHequation\newname
      }{}%
      \hyper@makecurrent{equation}%
      \Hy@raisedlink{\hyper@anchorstart{\@currentHref}{}\hyper@anchorend}
    \fi
  \fi
}
\makeatother
%%%%%

%%%%% RENEWING THE BIBLIOGRAPHY
\makeatletter
\renewenvironment{thebibliography}[1]
{%
% \section*{\refname}%
%\@mkboth{\MakeUppercase\refname}{\MakeUppercase\refname}%
\list{\@biblabel{\@arabic\c@enumiv}}%
{\settowidth\labelwidth{\@biblabel{#1}}%
\leftmargin\labelwidth
\advance\leftmargin\labelsep
\@openbib@code
\usecounter{enumiv}%
\let\p@enumiv\@empty
\renewcommand\theenumiv{\@arabic\c@enumiv}}%
\sloppy
\clubpenalty4000
\@clubpenalty \clubpenalty
\widowpenalty4000%
\sfcode`\.\@m}
{\def\@noitemerr
{\@latex@warning{Empty `thebibliography' environment}}%
\endlist}
\makeatother
%%%%%

\makeatletter
\define@key{Hyp}{raiselinks}[true]{%
  \lowercase{\Hyp@boolkey{#1}}{raiselinks}}
\makeatother

%%%%% CLEARDOUBLEPAGE TWEAK
%se si usa twoside con openright, il codice seguente non fa comparire 
%le intestazioni sulle pagine bianche alla fine del capitolo
\makeatletter
\def\cleardoublepage{\clearpage\if@twoside
\ifodd\c@page
\else\hbox{}\thispagestyle{empty}\newpage
\if@twocolumn\hbox{}\newpage\fi\fi\fi}
\makeatother
%%%%%
