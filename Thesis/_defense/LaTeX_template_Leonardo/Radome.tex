% $Header: /cvsroot/latex-beamer/latex-beamer/solutions/generic-talks/generic-ornate-15min-45min.en.tex,v 1.5 2007/01/28 20:48:23 tantau Exp $

\documentclass[trans,10pt]{beamer}

% This file is a solution template for:

% - Giving a talk on some subject.
% - The talk is between 15min and 45min long.
% - Style is ornate.

% Copyright 2004 by Till Tantau <tantau@users.sourceforge.net>.
%
% In principle, this file can be redistributed and/or modified under
% the terms of the GNU Public License, version 2.
%
% However, this file is supposed to be a template to be modified
% for your own needs. For this reason, if you use this file as a
% template and not specifically distribute it as part of a another
% package/program, I grant the extra permission to freely copy and
% modify this file as you see fit and even to delete this copyright
% notice. 

%%%%%%% per testo giustificato %%%%%%%%%%%%%%
\usepackage{ragged2e}
%%%%%%%%%%%%%%%%%%%%%
\usepackage[italian]{babel}
\usepackage[latin1]{inputenc}
\usepackage{times}
\usepackage[T1]{fontenc}
% Or whatever. Note that the encoding and the font should match. If T1
% does not look nice, try deleting the line with the fontenc.

\mode<presentation>
{
\usetheme{Malmoe}
%\usetheme{Berkeley}
%\usetheme{Warsaw}
%\usetheme{Copenhagen}
\usefonttheme{serif} 
\setbeamercovered{transparent}
  % or whatever (possibly just delete it)
\definecolor{miorosso}{RGB}{139,35,35}
\definecolor{miogrigio}{RGB}{229,229,229}
\setbeamercolor{item}{fg=miorosso}
\setbeamercolor{title}{fg=miorosso}
\setbeamercolor{frametitle}{fg=miorosso}
\setbeamercolor{frametitle}{bg=miogrigio}
\setbeamercolor{section in toc}{fg=miorosso}
\setbeamercolor{alerted text}{fg=miorosso}
\setbeamercolor{subsection in head/foot}{bg=miorosso}
\setbeamercolor{title in head/foot}{bg=miorosso}
%\setbeamerfont{subsection in head/foot}{size=\footnotesize}
%\setbeamerfont{section in head/foot}{size=\footnotesize}
%\setbeamerfont{title in head/foot}{size=\footnotesize}
%\setbeamerfont{author in head/foot}{size=\footnotesize}
\setbeamerfont{frametitle}{size=\normalsize}
\setbeamerfont{framesubtitle}{size=\small,shape=\itshape}
}


\title[I radome] % (optional, use only with long paper titles)
{I radome}

\subtitle
{Antenne II / Antenne in ambiente operativo} % (optional)

\author[L. Lucci - Universit� di Firenze] % (optional, use only with lots of authors)
%{F.~Author\inst{1} \and S.~Another\inst{2}}
{Leonardo Lucci \\
   {\scriptsize \textit{leonardo.lucci@unifi.it}}
}
% - Use the \inst{?} command only if the authors have different
%   affiliation.

\institute[Universit� di Firenze] % (optional, but mostly needed)
{
  Dipartimento di Elettronica e Telecomunicazioni\\
  \vspace*{6pt}Universit� di Firenze 
}
% - Use the \inst command only if there are several affiliations.
% - Keep it simple, no one is interested in your street address.

\date[] % (optional)
{A.A. 2008/2009}

\subject{Talks}
% This is only inserted into the PDF information catalog. Can be left
% out. 



% If you have a file called "university-logo-filename.xxx", where xxx
% is a graphic format that can be processed by latex or pdflatex,
% resp., then you can add a logo as follows:

 \pgfdeclareimage[height=0.8cm]{university-logo}{logo_1}
 \logo{\pgfuseimage{university-logo}}



% Delete this, if you do not want the table of contents to pop up at
% the beginning of each subsection:
\AtBeginSubsection[]
{
  \begin{frame}<beamer>{Outline}
    \tableofcontents[currentsection,currentsubsection]
  \end{frame}
}


% If you wish to uncover everything in a step-wise fashion, uncomment
% the following command: 

%\beamerdefaultoverlayspecification{<+->}


\begin{document}

\begin{frame}
  \titlepage
\end{frame}

\begin{frame}{Outline}
  \tableofcontents
  % You might wish to add the option [pausesections]
\end{frame}


% Since this a solution template for a generic talk, very little can
% be said about how it should be structured. However, the talk length
% of between 15min and 45min and the theme suggest that you stick to
% the following rules:  

% - Exactly two or three sections (other than the summary).
% - At *most* three subsections per section.
% - Talk about 30s to 2min per frame. So there should be between about
%   15 and 30 frames, all told.

%%%%%%%%%%%%%%%%%%%%%%%%%%%%%%%%%%%%%%%%%%%%
\section{Generalit�}

\begin{frame}{Funzione}{}

\begin{itemize}\justifying
	\item Il radome (radar dome) � l'involucro che in alcuni casi contiene l'antenna
	\item[]
	\item La funzione principale del radome � quella di proteggere l'antenna da esso racchiusa dagli agenti ambientali esterni (carico del vento, ghiaccio, neve, pioggia, temperatura)
	\item[]
	\item Idealmente il radome dovrebbe essere completamente trasparente alla radiofrequenza (RF) e non dovrebbe degradare le performance elettromagnetiche dell'antenna
\end{itemize}
\end{frame}

\begin{frame}{Tipologia}{}

\begin{itemize}	
\item	Un guscio in materiale dielettrico
\item[]
\item Strati di materiale metallico perforato
\item[]
\item Pu� comprendere strutture metalliche o dielettriche di sostegno
\end{itemize}

\end{frame}

\begin{frame}{Cenni storici}{}
\begin{itemize}\justifying
\item Gli aeromobili a bassa velocit� impiegati durante la seconda guerra mondiale, consentivano l'impiego di antenne esterne di tipo Yagi-Uda o ad array di dipoli operanti in VHF.
\item[]
\item Nel 1941 negli Stati Uniti e nel Regno Unito sono sperimentati radar a microonde e si ha il primo volo di un aeromobile ad alta velocit� che monta un radome in plexiglass per un radar sperimentale  in banda S (2-4GHz) della Western Electric
\end{itemize}
\end{frame}

\begin{frame}{Cenni storici}{}
\begin{itemize}\justifying
\item Dal 1943 iniziano a diffondersi radome per aeromobili in legno compensato (ampia diffusione anche a bordo di alcune imbarcazioni della marina statunitense, a bordo dei dirigibili e nelle stazioni di terra): problema dell'umidit� e della realizzazione di superfici con doppia curvatura
\item[]
\item Dal 1944 si iniziano a studiare strutture a sandwich a tre strati (skin-core-skin) con rivestimento in fibra di vetro e nucleo in polistirene
\end{itemize}
\end{frame}

\begin{frame}{Applicazioni}{}
\begin{itemize}
\item Sistemi terrestri
\item[]
\item Sistemi marittimi
\item[]
\item Sistemi a bordo di aeromobili
\item[]
\item Sistemi missilistici
\end{itemize}
\end{frame}

\begin{frame}{Applicazioni}{}
\begin{center}
\includegraphics[height=3.5cm]{Radome_satellitare}  \hspace{1cm} \includegraphics[height=3.5cm]{Radome_satellitare_imbarcazione}\\
Radome per comunicazioni satellitari: vista in primo piano  e installazione a bordo di una imbarcazione \\
\vspace*{1cm}{\small {[}D. J. Kozakoff, Analysis of Radome-Enclosed Antennas, Artech House, Boston-London, 1997{]} }
\end{center}
\end{frame}

\begin{frame}{Forme}{}
\begin{itemize}\justifying
\item Forma sferica (grandi antenne delle stazioni terrestri)
\item[]
\item forma smussata per contenere la resistenza aerodinamica o \emph{drag} (a bordo di velivoli o missili)
\item[]
\item spesso il radome � montato sul naso del velivolo per minimizzare il \emph{drag}
\end{itemize}
\end{frame}

\begin{frame}{Forme}{}
\begin{columns}
\column{.5\textwidth}\center
Radome al Misawa Security Operations Center, Misawa, Japan\\
\vspace*{.7cm}\includegraphics[width=\textwidth]{MisawaSecurityOperationCenter}
\column{.5\textwidth}\center
Radome montato sul \emph{cockpit} di un aereo\\
\includegraphics[width=\textwidth]{RadomeMeteoAereo1}\\
\vspace*{.3cm}\includegraphics[height=.4\textheight]{RadomeMeteoAereo}
\end{columns}
\end{frame}

\begin{frame}{Forme}{Esempi di radome a bordo di velivoli}
\begin{center}
\includegraphics[width=.6\textwidth]{AEWC} \\
Sistema Airborne Early Warning and Control (AEW\&C) a bordo di  un Boeing E-3 Sentry della Royal Air Force
\end{center}
\end{frame}


\begin{frame}{Materiali}{}
Fin dai primi anni gli studi sui materiali impiegati per la realizzazione dei radome hanno riguardato due aree principali:\vspace{12pt}
\begin{itemize}
\item radome ceramici impiegati in applicazioni missilistiche (altissima velocit�)
\item[]
\item strutture composite a sandwich che utilizzano materiali organici ad elevata resistenza
\end{itemize}
\end{frame}

\begin{frame}{Chi costruisce radome}{}
\begin{itemize}
	\item L-3 Communications ESSCO (Massachussets, USA / Ireland)\\
\textit{http://www.l-3com.com/ESSCO/index.html}
\item[]
\item AFC (Florida, USA)\\ 
\textit{http://www.afcsat.com/}
\item[]
\item Radome srl (Bergamo, Italy)\\
\textit{http://www.radome.it/home.html}
\item[]
\item Microwave Instrumentation Technologies (Georgia, USA) - ex Scientific Atlanta\\
\textit{http://www.mi-technologies.com/index.html}
\end{itemize}
\end{frame}

%%%%%%%%%%%%%%%%%%%%%%%%%%%%%%%%%%%%%%%%%%%%
\section{I materiali e le strutture}

\begin{frame}{Propriet� dei materiali}{}
Propriet� elettriche (dipendenti generalmente dalla frequenza di lavoro)
\begin{itemize}
	\item costante dielettrica relativa
\item tangente di perdita del dielettrico
\end{itemize}
\vspace*{12pt}Propriet� termo-meccaniche
\begin{itemize}
\item flessibilit�, durezza e resistenza
\item densit� del materiale
\item caratteristiche di assorbimento dell'acqua
\item resistenza all'erosione da pioggia
reazione agli shock termici
\end{itemize}
\end{frame}


\begin{frame}{Tipologia di materiali}{}
Per quanto attiene ai materiali i radome possono essere raggruppati in due grandi famiglie:\vspace{12pt}
\begin{itemize}\justifying
\item Radome con pareti dielettriche organiche : applicazioni a bassa temperatura (aeronautica civile e militare, veicoli terrestri e stazioni di terra) per cui T = 250�C (T = 500�C per brevi intervalli di tempo)
\item[]
\item Radome con pareti dielettriche inorganiche (ceramiche): applicazioni ad alta temperatura (aerei o missili supersonici)
\end{itemize}
\end{frame}

\subsection{Radome organici}

\begin{frame}{Strutture}{}
\begin{itemize}\justifying
\item strutture monolitiche (\textit{style-a} e \textit{style-b} [MIL-R-7705B]): si tratta di strutture solide costituite da resine che eventualmente  incorporano elementi di rinforzo (fibra di vetro)
\item[]
\item strutture \textcolor{red}{a sandwich (\textit{style-c} o \textit{A-sandwich}}, \textit{style-d} e \textit{style-e} [MIL-R-7705B]): si tratta di strutture che alternano strati di materiale ad alta densit� e costante dielettrica a strati di materiale a bassa densit� e costante dielettrica
\end{itemize}
\end{frame}

\begin{frame}{Struttura monolitica}{}
\end{frame}

\begin{frame}{Struttura a sandwich}{}
\end{frame}

\begin{frame}{Struttura a sandwich}{Laminato interno ed esterno}
	\end{frame}

\begin{frame}{Struttura a sandwich}{Nucleo}
\end{frame}

\subsection{Radome inorganici}

\begin{frame}{Radome inorganici}{}
\end{frame}

%%%%%%%%%%%%%%%%%%%%%%%%%%%%%%%%%%%%%%%%%%%%
\section{Interazione radome-antenna}

\begin{frame}{}{}
\end{frame}

\begin{frame}{Effetti del radome}{}
\end{frame}

\begin{frame}{Parametri che determinano le performance di un radome}{}
\end{frame}

\begin{frame}{Parametri di interesse nelle varie applicazioni}{}
\end{frame}

\begin{frame}{Parametri di interesse nelle varie applicazioni}{}
\end{frame}

%%%%%%%%%%%%%%%%%%%%%%%%%%%%%%%%%%%%%%%%%%%%
\section{Performance EM di un radome}

\subsection{Progetto EM di un radome}

\begin{frame}{Vincoli e parametri di progetto}{}
Le variabili che determinano i parametri di performance di un radome sono di due tipi:\\ \vspace{6pt}
\begin{itemize}\justifying
\item \textbf{Vincoli di progetto}: variabili che non costituiscono gradi di libert� per il progettista elettromagnetico, poich� vengono scelti per soddisfare esigenze irrinunciabili (requisiti termo-meccanici del materiale, resistenza aerodinamica o resistenza al vento della forma, dimensioni necessarie per proteggere l'antenna, caratteristiche meccaniche ed EM dell'antenna)
\item[]
\item \textbf{Variabili di progetto}: sono i gradi di libert� che il progettista ha per ottimizzare le performance del radome (in particolare composizione e configurazione delle pareti)
\end{itemize}
\end{frame}

\begin{frame}{Procedura di progetto}{}
\begin{itemize}\justifying
\item Il progetto di un radome inizia con la scelta di una struttura di base che viene ottimizzata mediante un procedimento euristico, di tipo \textit{trial-and-error}, in cui i parametri del radome vengono ottimizzati per successive valutazioni delle performance.
\item[]
\item La caratterizzazione del radome pu� essere effettuata sia mediante misure, che mediante l'ausilio di tecniche numeriche.
\item[]
\item In fase di progetto la conoscenza dei contributi di campo all'interno di un radome aiuta ad individuare le regioni che generano significative riflessioni interne e che contribuiscono maggiormente alla degradazione delle performance dell'antenna.
\end{itemize}
\end{frame}

\subsection{Caratterizzazione EM di un radome}

\begin{frame}{Misura della distribuzione di campo all'interno del radome}{Setup di misura}
\begin{center}
\includegraphics[width=.7\textwidth]{setup_misura} \\
\end{center}
\end{frame}

\begin{frame}{Misura della distribuzione di campo all'interno del radome}{Esempio di misura di side lobe level}
\begin{columns}
\column{.5\textwidth}\center
\includegraphics[width=.7\textwidth]{angolo_gimbal}\\
\vspace*{.2cm}\includegraphics[width=\textwidth]{misura_sll}
\column{.5\textwidth}
$f=16.5$GHz\\
$\beta=36�$ (angolo di \textit{gimbal})\\
$D=33$cm\\
$L=66$cm\\
$t=0.49$cm\\
$\epsilon=4$\\
\vspace*{12pt}\textcolor{red}{Le riflessioni dovute al radome determinano l'innalzamento dei lobi laterali}
\end{columns}
\end{frame}

\begin{frame}{Misura della distribuzione di campo all'interno del radome}{Esempio di misura dello sfasamento}
\begin{columns}
\column{.5\textwidth}\center
\includegraphics[width=.7\textwidth]{angolo_gimbal}\\
\vspace*{.2cm}\includegraphics[width=.8\textwidth]{misura_fase}
\column{.5\textwidth}
$f=16.5$GHz\\
$\beta=36�$ (angolo di \textit{gimbal})\\
$D=33$cm\\
$L=66$cm\\
$t=0.49$cm\\
$\epsilon=4$\\
\vspace*{12pt}\textcolor{red}{La distorsione del fronte di fase determina l'errore di boresight}
\end{columns}
\end{frame}

\begin{frame}{Misura della distribuzione di campo all'interno del radome}{Esempio di compact range per la misura delle performance EM di un sistema di antenna con radome}
\begin{center}
\includegraphics[width=.7\textwidth]{compact_range} \\
\vspace*{6pt}Naval Air Weapons Station Point Mugu \\(Scientific-Atlanta compact range 46m x 46m x 18m)
\end{center}
\end{frame}

\begin{frame}{Propagazione attraverso dielettrici piani multistrato}{}
\begin{itemize}
	\item Propagazione di un'onda piana attraverso un dielettrico piano multistrato indefinito
\item Si considera un modello a piastre piane (radome localmente piano)
\item Valutazione dell'angolo di incidenza
\end{itemize}
\begin{columns}
\column{.5\textwidth}\center
\includegraphics[width=.9\textwidth]{spm}
\column{.5\textwidth}
\includegraphics[width=.9\textwidth]{incidenza_su_piastra}
\end{columns}
\end{frame}

\begin{frame}{Propagazione attraverso dielettrici piani multistrato}{}
\begin{center}
Si suppone che il mezzo (0) sia il vuoto\\
\vspace*{6pt}\includegraphics[width=.7\textwidth]{multistrato_1} \\
\end{center}
\end{frame}

\begin{frame}{Propagazione attraverso dielettrici piani multistrato}{}
\begin{columns}
\column{.5\textwidth}\center
\includegraphics[width=\textwidth]{multistrato_2}
\column{.5\textwidth}
$n_i=\sqrt{\epsilon_{ri}}$ \\
\vspace*{6pt}indice di rifrazione del materiale (dielettrico)
\end{columns}
\vspace*{6pt}\begin{center}
Legge della rifrazione dell'ottica (Legge di Snell)\\
\vspace*{6pt}$n_0 \sin\theta_0=n_1 \sin\theta_1=n_2 \sin\theta_2=n_3 \sin\theta_3=n_4 \sin\theta_4$
\end{center}
\end{frame}

\begin{frame}{Propagazione attraverso dielettrici piani multistrato}{Problema di base:singola interfaccia (onde riferite immediatamente a sx e a dx dell'interfaccia)}
\begin{columns}
\column{.6\textwidth}\center
\includegraphics[height=3cm]{piano_incidenza}
\column{.4\textwidth}\center
\includegraphics[height=3cm]{singola_interfaccia}
\end{columns}
\begin{displaymath}
\begin{array}{c}
k_i^\perp=k_i^\parallel=k_0n_i\cos\theta_i\\
 \\
\begin{array}{ll}
Z_i^\perp=\frac{\zeta_i}{\cos\theta_i} & Z_i^\parallel=\zeta_i\cos\theta_i \\
 \\
R_1^\perp=\frac{B_1}{C_1}\arrowvert_{B_2=0}=\frac{Z_2^\perp-Z_1^\perp}{Z_2^\perp+Z_1^\perp} & R_1^\parallel=\frac{B_1}{C_1}\arrowvert_{B_2=0}=\frac{Z_2^\parallel-Z_1^\parallel}{Z_2^\parallel+Z_1^\parallel} \\
 \\
T_1^\perp=\frac{C_2}{C_1}\arrowvert_{B_2=0}=1+R_1^\perp & T_1^\parallel=\frac{C_2}{C_1}\arrowvert_{B_2=0}=(1+R_1^\parallel)\frac{\zeta_2}{\zeta_1}
\end{array}
\end{array}
\end{displaymath}
\end{frame}

\begin{frame}{Propagazione attraverso dielettrici piani multistrato}{}
\begin{columns}
\column{.5\textwidth}\center
Per la polarizzazione perpendicolare\\
\begin{displaymath}
\begin{array}{l}
C_2=C_1T_1+B_2R'_1\\
\\
B_1=C_1R_1+B_2T'_1\\
\\
R'_1=-R_1\\
\\
T'_1=1+R'_1=1-R_1
\end{array}
\end{displaymath}
\column{.5\textwidth}\center
Per la polarizzazione perpendicolare\\
\begin{displaymath}
\begin{array}{l}
C_2=C_1T_1+B_2R'_1\\
\\
B_1=C_1R_1+B_2T'_1\\
\\
R'_1=-R_1\\
\\
T'_1=(1+R'_1)\frac{\zeta_1}{\zeta_2}=(1-R_1)\frac{\zeta_1}{\zeta_2}
\end{array}
\end{displaymath}
\end{columns}
\end{frame}


\begin{frame}{Propagazione attraverso dielettrici piani multistrato}{}
\begin{center}
Per entrambi le polarizzazioni vale dunque:\\
\begin{displaymath}
\begin{array}{l}
C_1=\frac{1}{T_1}C_2+\frac{R_1}{T_1}B_2\\
 \\
B_1=\frac{R_1}{T_1}C_2+\frac{1}{T_1}B_2
\end{array}
\end{displaymath}\\
Ovvero in forma matriciale:\\
\begin{displaymath}
\left [
\begin{array}{c}
C_1\\B_1
\end{array}
\right ]
=\frac{1}{T_1}
\left [
\begin{array}{cc}
1 & R_1\\R_1 & 1
\end{array}
\right ]
\left [
\begin{array}{c}
C_2\\B_2
\end{array}
\right ]
\end{displaymath}
\end{center}
\end{frame}

\begin{frame}{Propagazione attraverso dielettrici piani multistrato}{Cascata di problemi elementari (onde riferite immediatamente a sx di ciascuna interfaccia)}
\begin{columns}
\column{.6\textwidth}\center
\includegraphics[height=3cm]{piano_incidenza}
\column{.4\textwidth}\center
\includegraphics[height=3cm]{singola_interfaccia}
\end{columns}
\begin{displaymath}
\left [
\begin{array}{c}
C_1\\B_1
\end{array}
\right ]
=\prod_{i=1}^N\frac{1}{T_1}
\left [
\begin{array}{cc}
e^{jk_it_i} & R_ie^{-jk_it_i}\\R_ie^{jk_it_i} & e^{-jk_it_i} 
\end{array}
\right ]
\left [
\begin{array}{c}
C_{N+1}\\B_{N+1}
\end{array}
\right ]=
\left [
\begin{array}{cc}
A_{11} & A_{12}\\A_{21} & A_{22} 
\end{array}
\right ]
\left [
\begin{array}{c}
C_{N+1}\\B_{N+1}
\end{array}
\right ]
\end{displaymath}
\end{frame}

\begin{frame}{Trasmissione attraverso una struttura A-sandwich}{Polarizzazione perpendicolare}
\end{frame}

\begin{frame}{Trasmissione attraverso una struttura A-sandwich}{Polarizzazione parallela}
\end{frame}

\begin{frame}{Tecniche di analisi}{Ottica Geometrica}
\end{frame}

\begin{frame}{Tecniche di analisi}{Ottica Geometrica in ricezione}
\end{frame}

\begin{frame}{Tecniche di analisi}{Ottica Geometrica in trasmissione}
\end{frame}

\begin{frame}{Tecniche di analisi}{Altre tecniche}
\end{frame}

\end{document}